\documentclass[a4paper, 11pt]{article}
\begin{document}

\section{Introduction}

\subsection{Reefer Container}

A reefer container, short for refrigirated container, is essencially a big rectangular container, equipped with a refrigeration unit at one end, opposite to its door. This containers, that are generally either 20 or 40 foot long, are used to transport temperature controlled cargo at long distance and can be set to mantain a specific temperature or pre-programmed to follow a determined temperature sequence.

These containers are formed by a steel frame and its walls are composed by alluminium sheets with foam insulation between them. The alluminium floors in T shape are responsible for the cool air supply to the cargo. This air, after cooling the cargo, thus getting warmer by receiving heat from the cargo and outside environment, ascends and is then conducted to the refregiration unit that, though the evaporator, cools it and sends it back to the cargo. 

Although this project focuses on temperature control, these containers are capable or regulating not only temperature, but also atmosphere (levels of CO$_2$ and O$_2$) and humidity.  

\end{document}
